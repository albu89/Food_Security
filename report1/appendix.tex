\chapter{Data}

\section{Crowd Flower}

Please categorize terms appearing in tweets about food in order to help us quantify users perception of \textbf{\emph{food cost}}, \textbf{\emph{food supply}}, \textbf{\emph{food needs}} and \textbf{\emph{food poverty}}. Overlaps may occur i.e. a term can potentially belong to more then one category. Such keywords should always be classified as B ) Likely. 


\subsection{Price}

Is the word or pair of words likely 1.) to be indicative of a user perception of food cost or 2.) can the word or pair of words be a synonym of the word price?

A. YES, the term is indicative of food cost and/or can be used as a synonym of price

\begin{itemize}

  \item pricy (quantification)
  \item expensive (quantification)
  \item cheap (quantification) 
  \item affordable (quantification)
  \item bill (synonym)
  \item receipt (synonym) 
  \item cost (synonym)
\end{itemize}


B. LIKELY, the term might be indicative of food cost and/or can be likely used as a synonym of price

\begin{itemize}
  \item low (quantification)
  \item high (quantification)
  \item increasing (quantification)
  \item ticket (synonym)
  \item quotation (synonym)
\end{itemize}


C. NO, the term is unlikely to be indicative of food cost and is neither a synonym of price

\begin{itemize}
  \item when
  \item chair
  \item boy
 
\end{itemize}


D. Not in English, not understandable, other issues.


\subsection{Supply}

Is the word or pair of words likely 1.) to be indicative of a user perception of food supply or 2.) can the word or pair of words be a synonym of the word supply?



A. YES, the term is indicative of food supply and/or can be used as a synonym of supply


\begin{itemize}
  \item available (quantification)
  \item accessible (quantification)
    \item lack (quantificaiton)
  \item amount (synonym)
  \item number (synonym)
   \item stock (synonym)
   \item ressource (synonym)
\end{itemize}

B. LIKELY, the term might be indicative of food supply and/or can be likely used as a synonym of supply

\begin{itemize}
  \item low (quantification)
  \item high (quantification)
  \item increasing (quantification)
  \item offer (synonym)
  \item quotation (synonym)
\end{itemize}

C. NO, the term is unlikely to be indicative of food supply and is neither a synonym of supply

\begin{itemize}
  \item when
  \item chair
  \item boy
\end{itemize}

D. Not in English, not understandable, other issues.



\subsection{Poverty}

Is the word or pair of words likely to be indicative of a user perception of food poverty or the user perception of wealth? 


A. YES, the term is indicative of food poverty or wealth

\begin{itemize}

  \item starving
  \item donation
  \item wealth 
  \item luxury
  \item profit
  \item help
  \item diabetes
  \item obesity
  \item healthy

\end{itemize}

B. LIKELY, the term might be indicative of food poverty and wealth\begin{itemize}
  \item lack
  \item excess
  \item plenty
  \item health 
  \item project
\end{itemize}

C. NO, the term is unlikely to be indicative of food supply and is neither a synonym of supply

\begin{itemize}
  \item when
  \item chair
  \item boy
\end{itemize}

D. Not in English, not understandable, other issues.



\subsection{Needs}

Is the word or pair of words likely to be indicative of a user perception of food needs

A. YES, the term is indicative of food needs

\begin{itemize}
  \item love
  \item want
  \item hate
  \item favorite 
  \item satisfied
  \item foodporn
  \item demand


\end{itemize}

B. LIKELY, the term might be indicative of food needs
\begin{itemize}
  \item crave
  \item urgent
  \item must
\end{itemize}

C. NO, the term is unlikely to be indicative of food needs 
\begin{itemize}
  \item when
  \item chair
  \item boy
\end{itemize}

D. Not in English, not understandable, other issues.

